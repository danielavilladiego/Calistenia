\documentclass{article}
\usepackage[utf8]{inputenc}
\usepackage[spanish]{babel}
\usepackage{listings}
\usepackage{graphicx}
\graphicspath{ {images/} }
\usepackage{cite}

\begin{document}

\begin{titlepage}
    \begin{center}
        \vspace*{1cm}
            
        \Huge
        \textbf{CALISTENIA}
            
        \vspace{0.5cm}
        \LARGE
        PARCIAL 1
            
        \vspace{1.5cm}
            
        \textbf{Daniela Rosa Villadiego Padilla}
            
        \vfill
            
        \vspace{0.8cm}
            
        \Large
        Despartamento de Ingeniería Electrónica y Telecomunicaciones\\
        Universidad de Antioquia\\
        Medellín\\
        Marzo de 2021
            
    \end{center}
\end{titlepage}

\tableofcontents

\section{Sección introductoria}
 Se le dará solución al desafío de describir como llevar unos objetos de una posición A a una posición B. Los objetos para realizar el ejercicio constan de una hoja de papel y dos tarjetas pequeñas del mismo tamaño.

\section{Sección de contenido} \label{contenido}

    \textbf{Los pasos para realizar el ejercicio son:}
    
     \vspace{0.3cm}
     
     Paso 1: Poner las tarjetas encima de la hoja blanca, las tarjetas deben estar la una encima de la otra casando perfectamente, todo el ejercicio debe hacerse con una sola mano, la que usted prefiera.
    
     \vspace{0.3cm}
     
     Paso 2: Teniendo las tarjetas una encima de la otra, lo que sigue es ponerlas de forma horizontal frente a usted, uniéndolas por el lado ancho (el lado más corto) una seguida de la otra manteniéndo la horizontalidad.
     
     \vspace{0.3cm}
      
     Paso 3: Tomar una de las tarjetas y montarla 1/4 (un cuarto) encima de la otra.
      
     \vspace{0.3cm}
     
     Paso 4: Con los dedos de la mano que crea necesarios levante las tarjetas por la unión, lentamente, tratando así de formar una pirámide con las dos tarjetas.  
     
      \vspace{1.5cm}
      
\section{Conclusión} 

    Siguiendo los anteriores pasos podrá llegar a una solución al ejercicio, la cual tal vez no sea la forma más fácil de hacerlo, pero entre otras cosas se está poniéndo a prueba su compresión lectora y su paciencia.





\end{document}
